\subsection{Kalibrierung}
Um ein hochwertiges Bild mit dem Teleskop aufzunehmen, müssen verschiedene Kalibrierungen durchgeführt werden.

Der \textit{dark frame} gleicht das thermische Rauschen der CCD aus.
Durch eine gewisse thermische Energie der Elektronen, können diese aus dem Valenzband in das Leitungsband gehoben werden und dadurch in einen Potentialtopf der CCD gelangen.
Um diese gemessenen Elektronen von dem eigentlichen light frame abzuziehen, wird ein Bild mit vollständig abgedunkelter CCD durchgeführt.
Da dieses Rauschen im Mittel steigt, ist die Integrationszeit des dark frames gleich der Integrationszeit des light frames.

Der \textit{Kalibrierungsframe} dient der Zuordnung der Wellenlänge zu den Pixeln.
Mit einer Gasentladungslampe eines bekannten Gases wird ein Spektrum aufgenommen, um die Position der Emissionslinien (also den Wellenlängen) der Position der Pixel zuzuordnen.

\iffalse
Der \textit{FLAT frame} dient zur Korrektur der Pixelempfindlichkeit auf der CCD.
Da nicht jeder Pixel die gleiche Lichtempfindlichkeit aufweist und diese durch Staub oder Unreinheiten stark beeinträchtigt wird, wird ein frame mit vollständig ausgeleuchteter CCD aufgenommen.
Dieser frame wird durch den -- bereits mit dem dark frame korrigierten -- light frame geteilt, um jeden Pixel auf seine Empfindlichkeit zu normieren.
\fi
