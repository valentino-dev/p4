\subsection{Theoretischer Hintergrund: Wasserstoff}
Die H$\alpha $ oder die Balmer--$\alpha $ Linie von Wasserstoff beschreibt den Übergang eines Elektrons in Wasserstoff von $n=3$ auf $n=2$.
Mit Hilfe der \textsc{Saha}--Gleichung kann die Besetzungszahl der Elektronen im $n=2$ Zustand geschrieben werden als
\begin{align} 
  n_2=\dfrac{8p_e\lambda _{\text{th,e}}^3\text{e}^{\chi /4k_BT}}{4k_BT+p_e\lambda _{\text{th,e}}^3Z_{\text{int,0}}}\\
  Z_{\text{int,0}}\approx 2\text{e}^{\chi /2k_BT}+8\text{e}^{\chi /4k_BT}+\hdots 
.\end{align} 
Dies ist genau der Anteil an Elektronen im Wasserstoff, der die H$\alpha $ Linie erzeugt.
Für eine detailiertere Herleitung sei auf \cite{anleitung464} und \cite{saha} verwiesen.
Die Größe $p_e=\SI{10}{Pa}$ ist der Elektronendruck, deren Wert experimentell überprüft wird.
