\subsection{Theoretischer Hintergrund: Spektralklassen}
Ein Stern durchläuft in seiner Lebenszeit auf der Hauptreihe verschiedene Fusionsprozesse.
Dabei fusioniert dieser ausgehend von Wasserstoff zu immer schwereren Atomen, wodurch das Alter eines Sterns anhand seiner Zusammensetzung gedeutet werden kann.
Aufschluss auf diese Zusammensetzung bietet eine spektralanalyse des Emissionsspektrums.
Da verschieden schwere Atome Licht von unterschiedlichen Wellenlängen emittieren, kann mit Hilfe eines Absorptionsspektrums (\ref{fig:spektralklasse}) oder Emissionsspektrums auf die Zusammensetzung eines Sterns und damit auf das Alter oder die Temperatur geschlossen werden.
