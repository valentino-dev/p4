\subsection{Durchführung \& Auswertung: P Cygni Profil\\Radialgeschwindigkeit}
Die Radialgeschwindigkeit des Sterns zum Beobachter kann mit der Rotverschiebung der H$\alpha $ Linie, also dem Emissionspeak des Sternwindes der H$\alpha $ Linie, berechnet werden.
Diese ergibt sich dann zu
\begin{align} 
  v_R=-\SI{87325+-2289}{m/s}
.\end{align}
%TODO herleitung v_R (formel)
Der Literaturwert liegt bei\cite{pcygniRadialvelocity}
\begin{align} 
  v_\infty^\text{lit}=\SI{-8.9}{km/s}
.\end{align} 
%TODO v_inf richtig?
Die Abweichung beträgt ca.\ $38.1\sigma $ bei einem relativen Fehler von 38.1\%. % TODO: der Relative fehler kann nicht stimmen. 2e3 von 8e4 sind nichtmal 10%.
Dies ist nicht mit dem Literaturwert vereinbar.
Eine signifikante Abweichung kann durch die Schwierigkeit des Ablesens des Emissionspeaks im P Cygni Spektrum begründet werden.
Da nicht viele Datenpunkte zur verfügung standen, ist hier ein exaktes Ablesen nicht möglich. % TODO: das wurde von der Halpha linie abgelesen, bei dem viele Punkte zur verfügung standen. Also das stimmmt so nicht. tbh weiß ich nicht wo der Fehler liegt, da die Daten gut sind und diese Abweichung nicht nur durch schlechtes Fitten oder Abglesen passieren kann. Ich glaube es wäre nochmal gut zu erwähnen, welche daten wir verwednet haben. Es könnte einfach sein, dass wir das falsche Kalibrationsspektrum haben.
