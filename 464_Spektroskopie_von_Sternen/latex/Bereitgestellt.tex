\subsubsection{Bereitgestellt: HD109358(G0), HD134083(F5), HD21428(B3), HD29488(A5), HD60179(A1)}
Da die aufgenommenen Sternspektren allein nicht für die Auswertung reichen, werden Daten von weiteren Sternen bereitgestellt.
Diese sind HD109358(G0), HD134083(F5), HD21428(B3), HD29488(A5) und HD60179(A1).
Die Analyse des Spektrums wurde mit nEDAR durchgeführt, kalibriert und dann mit einer \textsc{Gauss}--Kurve die H$\alpha $ Linie angepasst.
Für einen Stern sind diese Plots in Abb.\ (\ref{fig:1a}) (nEDAR), (\ref{fig:1b}) (Anpassung H$\alpha $) und (\ref{fig:1c}) (Kalibrierung) zu sehen.
Für die anderen vier Sterne ist dies analog; Abb.\ (\ref{fig:2a}), (\ref{fig:2b}), (\ref{fig:2c}) und (\ref{fig:3a}), (\ref{fig:3b}), (\ref{fig:3c}) und (\ref{fig:4a}), (\ref{fig:4b}), (\ref{fig:4c}) und (\ref{fig:5a}), (\ref{fig:5b}), (\ref{fig:5c}).

Hieraus ergeben sich
\begin{table}[h]
  \begin{tabular}{cc}
    \toprule
    Stern & Äquivalenzbreite H$\alpha $\\
    \midrule
    HD109358 & $\SI{2.066+-0.137e-10}{m}$ \\
    HD134083 & $\SI{3.268+-0.275e-10}{m}$ \\
    HD21428 & $\SI{4.300+-0.278e-10}{m}$ \\
    HD29488 & $\SI{7.446+-0.316e-10}{m}$ \\
    HD60179 & $\SI{9.154+-0.563e-10}{m}$ \\
    \bottomrule
  \end{tabular}
\end{table}
Diese sind gegen die Temperatur des jeweiligen Sterns in Abb.\ (\ref{fig:äquivalenzbreiten}) aufgetragen.
Die Temperatur ergibt sich aus der Spektralklasse.
Man siehe dafür \cite{anleitung464} oder Abb.\ (\ref{fig:spektralklasse}).
In diese Abbildung ist weiter die Besetzungszahl $n_2$ des Wasserstoffs eingezeichnet, die sich wie Gleichung (\ref{eq:n2}) verhält.

Zu erkennen ist, dass für tiefe Temperaturen bist zum Peak der Kurve bei $\SI{1e+4}{K}$ die Theorie sehr gut mit den Messungen übereinstimmt.
Die Datenpunkte liegen in diesem Intervall mit ihren Fehlergrenzen auf der Kurve.

Für Temperaturen größer als $\SI{1e+4}{K}$ gibt es zwar nur einen Datenpunkt, dieser liegt aber erheblich weit weg von der Kurve.
Für größere Temperaturen ist die Theorie nach dieser Messung als nicht mit der Realität vereinbar; dafür müsste die Kurve weniger stark abfallen und einen längeren Auslauf haben.
