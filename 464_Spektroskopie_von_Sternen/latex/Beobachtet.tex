\subsubsection{Beobachtet: HD32630(B3), HD56537(A3), HD77327(A1)} \label{sec:beobachtet}
Die beobachteten Sterne waren HD32630(B3), HD56537(A3) und HD77327(A1).
Die Maßnahmen für eine gute Qualität der Messung, wie in (\ref{sec:d-a-sternspektren}) beschrieben, wurden für diese Sterne befolgt.
Allerdings ohne Erfolg, da sich auf den ausgewerteten Spektren keine Emissionslinien erkennen lassen.
Dafür betrachtet man die Analyse des selbst aufgenommenen Spektrums aus Abb.\ (\ref{fig:beob_hd32630}).

Die Deutung dieses Spektrums beläuft sich darauf, dass eine breit gefächerte Menge an verschiedenen Lichtquellen (Wellenlängen) mit in das Spektrum eingeflossen sind.
Darunter ist wahrscheinlich das Spektrum der Sonne aufgrund des Vollmonds und durch die Streuung in den Wolken, aber auch das Licht der Städte am Horizont, welches sich über den Himmel verteilt. 
Hier lassen sich allerdings keine quantitativen Aussagen über die eingestreuten Wellenlängen treffen, da dieses Spektrum nicht eine definierte Emissionslinie aufweist.
