\section{Fazit}
Mit der Analyse der H$\alpha $ Linie wurden die Äquivalenzbreiten der bereitgestellten Daten verschiedener Sterne in Tab.\ (\ref{tab:äquivalenzbreiten}) berechnet.
Diese Äquivalenzbreiten wurden gegen die Temperatur in Abb.\ (\ref{fig:äquivalenzbreiten}) aufgetragen, um den theoretischen Verlauf von Gl.\ (\ref{eq:saha}) zu überprüfen.
Diese Gleichung ist mit einem Druck von $p_\text{e}=\SI{10}{Pa}$ und $\chi =\SI{13.54}{eV}$ aufgetragen.
Vergleicht man die Messung mit der Theorie, dann erkennt man, dass die Kurve nach ihrem Maximum nicht exakt auf die Daten passt.
Vor Allem ist der Auslauf der Kurve zu steil, daher erkennt man, dass der Druck auf einen größeren Wert korrigiert werden muss.

Der Sternwind oder das P Cygni Profil wurden anhand bereitgestellter Daten von P Cygni analysiert.
Es wurde mit Hilfe der Differenz von blauer Kante und Emissionspeak aus Tab. (\ref{tab:vinf}) die Endwindgeschwindigkeit berechnet.
Diese liegt, vereinbar mit der Theorie, unter $\SI{2000}{km/s}$.

Mit der Rotverschiebung des Emissionspeak wurde die Radialgeschwindigkeit von P Cygni auf einen Wert von $v_R=\SI{-87325+-2289}{m/s}$ bestimmt.
Bei einem relativen Fehler von $2,6\%$ liegt die Abweichung bei $38,1\sigma $.
Dieser Wert ist also nicht mit der Literatur vereinbar.
Mögliche Fehlerquellen wurden bereits in (\ref{sec:D-A-vR}) diskutiert.
