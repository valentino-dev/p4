\section{Einleitung}
Sterne lassen sich anhand ihrer Spektrallinien klassifizieren.
Eine prägnante Spektrallinie ist die H--$\alpha $ Linie.
Ihre Breite gibt Auskunft über die Größe der Besetzungszahl der Wasserstoffatome in einem Stern.
Mit dieser Information lassen sich Aussagen über die Beschaffenheit eines Sterns tätigen.

Sternwinde beschreiben die Abstoßung von Gas aus der äußersten Hülle eines Sterns aufgrund von Fusionsdruck.
Für verschiedene Spektralklassen unterscheiden sich Sternwinde nicht nur in ihrer Masse und Geschwindigkeit, sondern auch durch ihre Ursache.
