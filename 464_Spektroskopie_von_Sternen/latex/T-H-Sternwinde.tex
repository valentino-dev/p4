\subsection{Theoretischer Hintergrund: Sternwinde}
Überwiegt der Strahlungsdruck dem Gravitationsdruck in der oberen Atmosphäre von Sternen, so stößt dieser seine äußere Hülle ab.
Dabei sind zwei verschiedene Sternwinde zu unterscheiden.

Sterne, wie Rote Riesen oder Überriesen die nicht mehr auf der Hauptreihe sind, stoßen oft viel Masse $\left(\dot{M}>10^{-3}\text{M}_\odot/\text{y}\right)$ mit langsamer Geschwindigkeit $\left(v=\SI{10}{km/s}\right)$ aus.
Diese Winde sind von Strahlungsdruck angetrieben.

Sterne, wie O und B Sterne, stoßen nur wenig Masse $\left(\dot{M}<10^{-6}\text{M}_\odot/\text{y}\right)$ mit großer Geschwindigkeit $\left(v>\SI{1}{km/s}-\SI{2000}{km/s}\right)$ ab.
Diese großen Geschwindigkeiten kommen von Resonanzen der Strahlung des Sterns mit dem Material des Sternwindes.
