\subsection{Durchführung \& Auswertung:\\Sternspektren} \label{sec:d-a-sternspektren}
Das aufgenommene Sternspektrum wird mit dem Programm nEDAR analysiert.
Mit dem Kalibrierbild können Wellenlängen zu den Pixeln zugeordnet werden.
Dabei wird von einem linearen Zusammenhang zwischen Wellenlänge und Pixel ausgegangen, sodass die Zuordnung über einen linearen Fit geschehen kann.

Die H$\alpha $ Linienbreite kann mit Hilfe eines Fits an das, mit nEDAR extrahierte, Spektrum berechnet werden.
Diese Linienbreite wird dann in Abhängigkeit der Temperatur aufgetragen und mit der Besetzungszahl $n_2\left(T\right)$ verglichen.

Die Auswertung der Daten hat gezeigt, dass die selbst aufgenommenen Spektren nicht verwertet werden können.
Grund hierfür waren die Umstände während der Messung.
Die Nacht war wegen des Vollmondes nicht vollständig dunkel, daher ist das Spektrum der Sonne in die Messung eingegangen.
Gegen Mitternacht zogen Wolken in großen Teilen des Himmels auf, was die Messung erschwert hat.
Zudem gab es viel Lichtverschmutzung bei Sternen, die vor Allem in der Nähe des Horizonts lagen.
Daher wurde sich auf die Messung von Sternen in der Nähe des Zenits in möglichst großer Entfernung von Mond und Wolken konzentriert.
