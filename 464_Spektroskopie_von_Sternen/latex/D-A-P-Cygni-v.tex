\subsection{Durchführung \& Auswertung: P Cygni Profil\\Endsternwindgeschwindigkeit}
Analog zur H$\alpha $ Linie wird das Spektrum von P Cygni aufgenommen, mit nEDAR analysiert und ein linearer Fit zur Wellenlängenkalibrierung durchgeführt.
Da es wie auch in (\ref{sec:beobachtet}) nicht möglich war ein gelungenes Spektrum aufzunehmen, wird hier auf die bereitgestellten Daten zugegriffen.
Hierfür vergleicht man die Analyse des selbst aufgenommenen Spektrums in Abb.\ (\ref{fig:beob_pcygni}) und die des bereitgestellten Spektrums in Abb.\ (\ref{fig:bereit_pcygni}).

Um die Endwindgeschwindigkeit $v_\infty$ zu bestimmen kann wie in \ref{sec:T-H-P-Cygni} vorgegangen werden.
Es ist
\begin{align} 
  v_\infty=\dfrac{c\left(\lambda _\text{e}^2-\lambda _\text{b}^2\right)}{\lambda _\text{e}^2+\lambda _\text{b}^2}
.\end{align} 
Die Wellenlängen $\lambda _\text{e}$ und $\lambda _\text{b}$ werden an dem Plot des P Cygni Profils abgelesen.
Dies wird für die Linien H$\alpha$, N\textsc{ii} $\SI{6482,0}{\r{A}}$ und He\textsc{ii} $\SI{6678,2}{\r{A}}$ durchgeführt.  
Diese sind als gesamtes Spektrum in Abb.\ (\ref{fig:pcygniSpektrum}) und einzeln in (\ref{fig:pcHe2}), (\ref{fig:pcH}), (\ref{fig:pcN2}) und (\ref{fig:pcSi3}) zu sehen.
Die blaue Kannte ist als blaue vertikale Linie eingezeichnet.
Es ergeben sich die Werte aus Tab.\ (\ref{tab:vinf}).
Das Mittel liegt bei
\begin{align} 
  \overline{v_\infty}\approx \SI{1.78+-3366e+5}{m/s}
.\end{align} 
Die Größenordnung ($\approx \SI{178}{km/s}$) liegt im erwarteten Bereich von bis zu $\SI{2000}{km/s}$.
Der relativ große Fehler auf die Endwindgeschwindigkeit der He\textsc{ii} Linie (und damit auch auf den Mittelwert) kommt von den wenigen Datenpunkten, an die eine \textsc{Gauss}--Kurve angepasst werden kann. 
