\RequirePackage{amsthm} %https://tex.stackexchange.com/questions/687324/unknown-theoremstyle-warning-with-springer-nature-template
\documentclass[sn-mathphys-num,iicol]{sn-jnl}

%\usepackage{sn-jnl.sty}
\usepackage{graphicx}%
\usepackage{multirow}%
\usepackage{amsmath,amssymb,amsfonts}%
\usepackage{amsthm}%
\usepackage{physics}
\usepackage{siunitx}
\usepackage{mathrsfs}%
\usepackage[title]{appendix}%
\usepackage{xcolor}%
\usepackage{textcomp}%
\usepackage{manyfoot}%
\usepackage{booktabs}%
\usepackage{algorithm}%
\usepackage{algorithmicx}%
\usepackage{algpseudocode}%
\usepackage{listings}%
\usepackage{newtxmath}%
\usepackage[tiny]{titlesec}%
\usepackage[ngerman]{babel}
\usepackage{booktabs}

\theoremstyle{thmstyleone}
\newtheorem{theorem}{Theorem}
\newtheorem{proposition}[theorem]{Proposition}

\theoremstyle{thmstyletwo}
\newtheorem{remark}{Remark}

\theoremstyle{thmstylethree}
\newtheorem{definition}{Definition}

\raggedbottom

\newcommand{\td}{\text{d}}

\titleformat{\subsection}{}{\thesubsection}{1em}{\itshape}
\titleformat{\subsubsection}{}{\thesubsubsection}{1em}{\itshape}

\begin{document}
        
\title[]{Praktikum 4 -- Versuch 422: Rastertunnelmikroskop}
\author*[1]{\fnm{Jonas} \sur{Wortmann}}\email{s02jwort@uni-bonn.de}
\author*[2]{\fnm{Angelo} \sur{Brade}}\email{s72abrad@uni-bonn.de}
\affil*[1,2]{Rheinische Friedrich--Wilhelms--Universität, Bonn}

\maketitle

\section{Einleitung}
Um atomare Auflösungen und mikroskopische Materialstrukturen zu erkennen, ging aus der Vermessung von Austritsarbeiten eine neue Technologie hervor, die Vergrößerungen jenseits der optischen Begrenzung bietet.
Wenn sich zwei leitende Materialien mit verschiedenen Austrittsarbeiten nahe genug kommen, dann ist es Elektronen im \textsc{Fermi}--Niveau möglich, zwischen den Materialien zu tunneln.
Das Rastertunnelmikroskop (STM, scanning tunneling microscope) verwendet diesen Effekt, um Strukturen von Materialien auf mikroskopischer Ebene aufzulösen.

\section{Experimenteller Aufbau}
Der experimentelle Aufbau ist in Abb.\ (\ref{fig:aufbau}) zu sehen.
Das verwendete STM ist das \glqq NaioSTM\grqq{} von \textit{Nanosurf} (\url{www.nanosurf.com}).
\begin{figure}[t]
  \centering
  \includegraphics[width=.5\textwidth]{422_aufbau.png}
  \caption{Experimentiertisch.\cite{anleitung422}} \label{fig:aufbau}
\end{figure}
Das STM ist in Abb.\ (\ref{fig:naiostm}) zu sehen.
Der silberne Zylinder ist der Probenhalter, an dem die Probe mit einem Magneten festgehalten wird.
Der Probenhalter wird mit dem Slip--Stick Mechanismus bewegt.
Gegenüber der Probe ist eine Klammer, die einen Pt--Ir Draht hält (Abb.\ (\ref{fig:naiostm_closeup})).
Dieser ist die Spitze des Mikroskops.
Siehe auch Abb.\ (\ref{fig:spitze1}) und (\ref{fig:spitze2}).
\begin{figure}[t]
  \centering
  \includegraphics[width=.5\textwidth]{422_naiostm.png}
  \caption{NaioSTM von Nanosurf.\cite{nanosurf}} \label{fig:naiostm}
\end{figure}
\begin{figure}[t]
  \centering
  \includegraphics[width=.5\textwidth]{422_naiostm_closeup.png}
  \caption{NaioSTM von Nanosurf, Nahaufnahme der Spitze und Probe.\cite{nanosurf}} \label{fig:naiostm_closeup}
\end{figure}
\subsection{Theoretischer Hintergrund: STM}
Werden Spitze und Probe nah genug aneinander gefahren (s. Abb.\ (\ref{fig:naiostm_closeup})\footnote{Während der Messung sind Spitze und Probe in einem Abstand von wenigen zehntel Nanometer.}), so ist es den Elektronen im \textsc{Fermi}--Niveau der Probe möglich zur Spitze zu tunneln.
Dadurch entsteht ein Tunnelstrom.
Ein Bild der Probe wird erzeugt, in dem der gemessene Strom gegen den Ort aufgetragen wird.
Dieses Bild wird mit der \glqq Easyscan 2\grqq{} Software von Nanosurf aufgenommen.

Die Spitze kann mit makroskopischen Methoden von einem langen Pt--Ir Draht gewonnen werden.
Dabei wird ein kleines Stück des Drahtes mit einer Zange abgerissen, wodurch sich eine mikroskopisch kleine Spitze bildet. Das ist das sog. Reissen einer Spitze.
Da die Spitze nicht zu allen Seiten streng monoton fallend sein muss, sondern ausreichend ist, wenn es nur eine Stelle gibt, die der Probe am nächsten ist, kann diese Methode gut verwendet werden. % TODO: streng monoton fallend hört sich etwas over the top an

Die Probe wird mit dem Slip--Stick Mechanismus an die Spitze herangefahren.
Dabei wird der Probenhalter mit einem anderen Material (z.B.\ Gummi) in Kontakt gebracht.
Da der Haftreibungskoeffizient größer als der Gleitreibungskoeffizient ist, kann durch langsame Bewegung des Gummis die Haftung zum Probenhalter erhalten bleiben und der Probenhalter wird verschoben.
Durch schnelles Zurückziehen des Gummis gleitet dies über den Probenhalter zurück in seine ursprüngliche Position und kann erneut am Probenhalter haften.

Die Bewegung der Probe orthogonal zur Spitze werden mit Hifle von \textsc{Piezo}--Kristallen realisiert.
Ein \textsc{Piezo}--Kristall besitzt in einer bestimmten Achse des Gitters keine Spiegelsymmetrie, wodurch mechanische Stauchung und Streckung eine Potentialdifferenz im Kristall verursachen.
Dieses Phänomen funktioniert auch andersherum. % TODO: kann man saver schöner schreiben
Druch eine angelegte Spannung verschieben sich die Ladungsträger, wodurch eine Stauchung oder Streckung verursacht wird.
Dadruch wird die Probe (der Probenhalter) orthogonal verschoben. % TODO: Warum 'orthogonal'?

Das STM kann in zwei Modi verwendet werden.

Im \textit{constant current mode} wird die Entfernung der Spitze von der Probe so reguliert, dass immer die gleiche Stromstärke gemessen wird.
Dies wird mit einem PID--Regelkreis (proportional integral differential) realisiert.
Die Einstellung hierfür sind $P=2000$, $I=2000$ und $D=0$. % TODO: P=1000
Dieser Modus ist vor Allem für Proben mit unebenem Höhenprofil vorteilhaft, um die Spitze nicht auf der Probe auflaufen zu lassen. % TODO: Geschwindigkeit diskutieren

Im \textit{constant height mode} wird die Entfernung der Spitze von der Probe konstant gelassen.
Dies ermöglicht eine wesentlich schnellere Messung, ist allerdings nur für ebene Proben ratsam.
Die Einstellung des PID--Regelkreises sind $P=0$, $I=4$ und $D=0$.

\section{Durchführung \& Auswertung:\\Goldprobe}
Die verwendete Goldprobe ist \glqq Gold14\grqq{} und in Abb.\ (\ref{fig:goldfern}), (\ref{fig:goldmittel}) und (\ref{fig:goldnah}) zu erkennen.
Die aufgenommenen Bilder sind (\ref{fig:g10nmc}), (\ref{fig:g10nmz}), (\ref{fig:g10nm50mVc}), (\ref{fig:g10nm50mVz}), (\ref{fig:g200nmc}), (\ref{fig:g200nmz}), (\ref{fig:g200nm50mVc}), (\ref{fig:g200nm50mVz}), (\ref{fig:g400nmc}) und (\ref{fig:g400nmz}).

In den makroskopischen Aufnahmen ist zu erkennen, dass die Goldprobe eine ebene Oberfläche besitzt, die durch viele Kratzer beschädigt ist.
Zudem sind drei prägnante Einkerbungen und ein runder Fleck(?) zu erkennen. % TODO: Fleck -> Eindruck
An diesen vier Stellen wurde die mikroskopische Untersuchung vermieden. % Das war uns egal lol, aber finde ich gut
Der Zusand dieser Probe ist für diesen Versuch föllig ausreichend. % Finde ich fr ein gutes Statement

\subsection{Untersuchung STM: \textit{constant current mode}}
Der \textit{constant current mode} wurde hier aufgrund der vielen Kratzer und Rauheit der Oberfläche verwendet, um die Spitze nicht zu beschädigen.

In Abb.\ (\ref{fig:g400nmc}) ist eine helle Wolke mit grauen Verschmierungen zu erkennen. 
Eine genauere Auflösung der Wolke ist in Abb.\ (\ref{fig:g200nmz}) und (\ref{fig:g200nm50mVz}), wobei die grauen Unterteilungen deutlich werden.
Dieser Bilder waren erwartet, da bei Gold das \textsc{Fermi}--Niveau im Leiterband ist und eine Auflösung des Gitters daher nicht möglich ist.
Abb.\ (\ref{fig:g10nmz}) und (\ref{fig:g10nm50mVz}) sind für die Auflösung der Wolken zu nah an der Probe.
Eine Aussage über die makroskopische Beschaffenheit der Oberfläche lässt sich hiermit nicht tätigen, allerdings kann erkannt werden, dass die Wolken ein Streifenmuster aufweisen.

Alle Stromkarten (\ref{fig:g10nmc}), (\ref{fig:g10nm50mVc}), (\ref{fig:g200nmc}), (\ref{fig:g200nm50mVc}) und (\ref{fig:g400nmc}) liefern identische Ergebnisse auf unterschiedlichen Größenskalen.
Es ist eine fast uniforme Stromverteilung zu erkennen mit vereinzelten sehr hellen Pixeln.
Dieses Ergebnis war erwartet, da hier im \textit{constant current mode} gearbeitet worden ist.

\section{Durchführung \& Auswertung:\\HOPG--Probe}
Die verwendete HOPG (highly oriented pyrolytic graphite) Probe ist \glqq HOPG8\grqq{} und in Abb.\ (\ref{fig:graphitfern}), (\ref{fig:graphitmittel}) und (\ref{fig:graphitnah}) zu erkennen.
Die aufgenommenen Bilder sind Abb.\ (\ref{fig:gr4nm50nVc}), (\ref{fig:gr4nm50nVz}), (\ref{fig:gr100nm100nVc}), (\ref{fig:gr100nm100nVz}), (\ref{fig:gr100nm200nVc}), (\ref{fig:gr100nm200nVz}), (\ref{fig:gr10nm200nVc}), (\ref{fig:gr10nm200nVz}), (\ref{fig:gr2nm50nVc}), (\ref{fig:gr2nm50nVz}), (\ref{fig:gr2nm50nVc2}), (\ref{fig:gr2nm50nVz2}), (\ref{fig:gr2nm50nVc3}) und (\ref{fig:gr2nm50nVz3}).

In den makroskopischen Aufnahmen ist eine äußerst unebene Struktur zu erkennen, welche aus prägnanten quaderförmigen Kristallstrukturen besteht. % TODO: Welche makrsoskopischen Aufnahmen genau? Ich kann da garkeite Struktur erkennen.
Aufgrund der natürlichen Unordnung und Unebenheit der Struktur sind keine Unreinheiten oder Beschädigungen zu erkennen.
Für die Untersuchung wurde eine möglichst flache Stelle auf der Probe verwendet.

\subsection{Untersuchung STM: constant current mode}
Der \textit{constant current mode} wurde hier aufgrund der makroskopischen Unebenheiten verwendet, um die Spitze nicht in die Probe zu bewegen.

Die Untersuchung der Probe -- vor Allem die Auflösung der Gittersturktur -- stellte sich als unerwartet schwierig dar, sodass diese nicht beobachtet werden konnte.
Alle Höhenkarten weisen Artefakte, Rauschen oder Schlieren auf; alle Stromkarten weisen vor Allem Rauschen und wenige Unebenheiten auf.

Um eine atomare Auflösung zu erzielen wurden folgende Vorgehensweisen versucht:
Eine neue Spitze wurde aus einer alten Spitze gerissen und mit großer Vorsicht in den Spitzenhalter eingebaut.
Dabei ist die Spitze nicht in Berührung mit anderen Gegenständen gekommen.
Die HOPG Probe wurde mehrfach abgezogen und der Probenhalter gereinigt.
Beim Anbringen der Probe wurde darauf geachtet, dass eine möglichst ebene Fläche der Probe zur Spitze zeigt.
Das Heranfahren an die Probe wurde nur auf eine noch makroskopisch unterschiedbare Entfernung mit der Hand getan, die weitere Bewegung wurde mit \textit{Approach} und \textit{Advance} vollführt, sodass es nicht möglich war, dass Probe und Spitze in Berührung gekommen sind.
Ein \textit{Adjust Slope} wurde durchgeführt und ein \textit{Tip Cleaning Pulse} gegeben.
Die Bilder wurden ohne Erfolg mit verschiedenen PID Parametern, verschiedenen Geschwindigkeiten und verschiedenen Größenordnungen gemacht.
Auch das Klopfen auf den Tisch hat nicht funktioniert.
Alle Schritte wurden für mehrere Spitzen ($<5$) durchgeführt und jedes mal sorgfältig gearbeitet.
Es konnte kein Bild in atmoarer Auflösung erzielt werden.

% TODO: Wir können noch passende Bilder an dieser Stelle referenzieren, um ein Beispiel zu geben, dass wir auf den passenden Größenskalen gearbeitet haben

% TODO: Allgemein finde ich das schon nicht schlecht. Aber ich habe das Gefühl, dass selbst die Bilder für die Goldprobe echt schlecht waren. Wenn ich mich an die Bilder erinnere, welche wir in dem Handbuch gesehen hatten, dann könnten sich gefühlt alle unsere Aufnahmen dort einreihen. Vielleicht sollte insbesondere bei der Goldprobe diese Möglichkeit nochmal erwähnt werden.

% TODO: Vielleicht sollte auch nochmal diskutiert werden, dass wir nur eine einzige constant hight Aufnahmen haben. Außerden könnte die auch nochmal inhaltlich diksutiert werden.
\section{Fazit}

\bibliography{refs}

\clearpage
\section{Appendix}
\begin{figure}[h]
  \centering
  \includegraphics[width=.5\textwidth]{../data/Spitze1_1.png}
  \caption{Spitze 1.} \label{fig:spitze1}
\end{figure}
\begin{figure}[h]
  \centering
  \includegraphics[width=.5\textwidth]{../data/Spitze1_2.png}
  \caption{Spitze 2.} \label{fig:spitze2}
\end{figure}
\clearpage
\begin{figure}[t]
  \centering
  \includegraphics[width=.5\textwidth]{../data/Gold_1.png}
  \caption{Gold fern.} \label{fig:goldfern}
\end{figure}
\begin{figure}[t]
  \centering
  \includegraphics[width=.5\textwidth]{../data/Gold_2.png}
  \caption{Gold mittel.} \label{fig:goldmittel}
\end{figure}
\begin{figure}[t]
  \centering
  \includegraphics[width=.5\textwidth]{../data/Gold_3.png}
  \caption{Gold nah.} \label{fig:goldnah}
\end{figure}
\begin{figure}[t]
  \centering
  \includegraphics[width=.5\textwidth]{../data/Graphit_3.png}
  \caption{Graphit fern.} \label{fig:graphitfern}
\end{figure}
\begin{figure}[t]
  \centering
  \includegraphics[width=.5\textwidth]{../data/Graphit_1.png}
  \caption{Graphit mittel.} \label{fig:graphitmittel}
\end{figure}
\begin{figure}[t]
  \centering
  \includegraphics[width=.5\textwidth]{../data/Graphit_2.png}
  \caption{Graphit nah.} \label{fig:graphitnah}
\end{figure}
\clearpage

% TODO: Grobe aufteilung. Kann immer geändert werden. Ist jetzt nicht so festgelegt. Nur als Orientierung.
\section{Goldprobe}
\begin{figure}[h]
        \centering
        \includegraphics[width=.5\textwidth]{../data/Gold_10nm_current.png}
        \caption{Gold: Stromkarte im const. current Modus (Setpoint=\SI{1}{\nano A}, P-Gain=\SI{1000}{}, I-Gain=\SI{2000}{} und Tip voltage=\SI{1}{V})} \label{fig:g10nmc}
\end{figure}
\begin{figure}[h]
        \centering
        \includegraphics[width=.5\textwidth]{../data/Gold_10nm_z.png}
        \caption{Gold: Höhenkarte im const. current Modus (Setpoint=\SI{1}{\nano A}, P-Gain=\SI{1000}{}, I-Gain=\SI{2000}{} und Tip voltage=\SI{1}{V})} \label{fig:g10nmz}
\end{figure}
\begin{figure}[h]
        \centering
        \includegraphics[width=.5\textwidth]{../data/Gold_10nm_50mV_current.png}
        \caption{Gold: Stromkarte im const. current Modus (Setpoint=\SI{1}{\nano A}, P-Gain=\SI{1000}{}, I-Gain=\SI{2000}{} und Tip voltage=\SI{50}{\milli V})} \label{fig:g10nm50mVc}
\end{figure}
\begin{figure}[h]
        \centering
        \includegraphics[width=.5\textwidth]{../data/Gold_10nm_50mV_z.png}
        \caption{Gold: Höhenkarte im const. current Modus (Setpoint=\SI{1}{\nano A}, P-Gain=\SI{1000}{}, I-Gain=\SI{2000}{} und Tip voltage=\SI{50}{\milli V})} \label{fig:g10nm50mVz}
\end{figure}
\begin{figure}[h]
        \centering
        \includegraphics[width=.5\textwidth]{../data/Gold_200nm_current.png}
        \caption{Gold: Stromkarte im const. current Modus (Setpoint=\SI{1}{\nano A}, P-Gain=\SI{1000}{}, I-Gain=\SI{2000}{} und Tip voltage=\SI{1}{V})} \label{fig:g200nmc}
\end{figure}
\begin{figure}[h]
        \centering
        \includegraphics[width=.5\textwidth]{../data/Gold_200nm_z.png}
        \caption{Gold: Höhenkarte im const. current Modus (Setpoint=\SI{1}{\nano A}, P-Gain=\SI{1000}{}, I-Gain=\SI{2000}{} und Tip voltage=\SI{1}{V})} \label{fig:g200nmz}
\end{figure}
\begin{figure}[h]
        \centering
        \includegraphics[width=.5\textwidth]{../data/Gold_200nm_50mV_current.png}
        \caption{Gold: Stromkarte im const. current Modus (Setpoint=\SI{1}{\nano A}, P-Gain=\SI{1000}{}, I-Gain=\SI{2000}{} und Tip voltage=\SI{50}{\milli V})} \label{fig:g200nm50mVc}
\end{figure}
\begin{figure}[h]
        \centering
        \includegraphics[width=.5\textwidth]{../data/Gold_200nm_50mV_z.png}
        \caption{Gold: Höhenkarte im const. current Modus (Setpoint=\SI{1}{\nano A}, P-Gain=\SI{1000}{}, I-Gain=\SI{2000}{} und Tip voltage=\SI{50}{\milli V})} \label{fig:g200nm50mVz}
\end{figure}
\begin{figure}[h]
        \centering
        \includegraphics[width=.5\textwidth]{../data/Gold_400nm_current.png}
        \caption{Gold: Stromkarte im const. current Modus (Setpoint=\SI{1}{\nano A}, P-Gain=\SI{1000}{}, I-Gain=\SI{2000}{} und Tip voltage=\SI{1}{V})} \label{fig:g400nmc}
\end{figure}
\begin{figure}[h]
        \centering
        \includegraphics[width=.5\textwidth]{../data/Gold_400nm_z.png}
        \caption{Gold: Höhenkarte im const. current Modus (Setpoint=\SI{1}{\nano A}, P-Gain=\SI{1000}{}, I-Gain=\SI{2000}{} und Tip voltage=\SI{1}{V})} \label{fig:g400nmz}
\end{figure}

\clearpage
\section{HOPG}
% TODO: Beachte, dass die beiden pngs hier mit 1V benannte sind, aber eigentlich 50mV Tip voltage hatten
\begin{figure}[h]
        \centering
        \includegraphics[width=.5\textwidth]{../data/Graphit_4nm_1V_current.png}
        \caption{HOPG: Stromkarte im const. current Modus (Setpoint=\SI{1}{\nano A}, P-Gain=\SI{1000}{}, I-Gain=\SI{2000}{} und Tip voltage=\SI{50}{\nano V})} \label{fig:gr4nm50nVc}
\end{figure}
\begin{figure}[h]
        \centering
        \includegraphics[width=.5\textwidth]{../data/Graphit_4nm_1V_z.png}
        \caption{HOPG: Höhenkarte im const. current Modus (Setpoint=\SI{1}{\nano A}, P-Gain=\SI{1000}{}, I-Gain=\SI{2000}{} und Tip voltage=\SI{50}{\nano V})} \label{fig:gr4nm50nVz}
\end{figure}
\begin{figure}[h]
        \centering
        \includegraphics[width=.5\textwidth]{../data/Graphit2_current.png}
        \caption{HOPG: Stromkarte im const. current Modus (Setpoint=\SI{1}{\nano A}, P-Gain=\SI{1000}{}, I-Gain=\SI{2000}{} und Tip voltage=\SI{100}{\nano V})} \label{fig:gr100nm100nVc}
\end{figure}
\begin{figure}[h]
        \centering
        \includegraphics[width=.5\textwidth]{../data/Graphit2_z.png}
        \caption{HOPG: Höhenkarte im const. current Modus (Setpoint=\SI{1}{\nano A}, P-Gain=\SI{1000}{}, I-Gain=\SI{2000}{} und Tip voltage=\SI{100}{\nano V})} \label{fig:gr100nm100nVz}
\end{figure}
\begin{figure}[h]
        \centering
        \includegraphics[width=.5\textwidth]{../data/Graphit3_current.png}
        \caption{HOPG: Stromkarte im const. current Modus (Setpoint=\SI{5}{\nano A}, P-Gain=\SI{1000}{}, I-Gain=\SI{2000}{} und Tip voltage=\SI{200}{\nano V})} \label{fig:gr100nm200nVc}
\end{figure}
\begin{figure}[h]
        \centering
        \includegraphics[width=.5\textwidth]{../data/Graphit3_z.png}
        \caption{HOPG: Höhenkarte im const. current Modus (Setpoint=\SI{5}{\nano A}, P-Gain=\SI{1000}{}, I-Gain=\SI{2000}{} und Tip voltage=\SI{200}{\nano V})} \label{fig:gr100nm200nVz}
\end{figure}
\begin{figure}[h]
        \centering
        \includegraphics[width=.5\textwidth]{../data/Graphit4_current.png}
        \caption{HOPG: Stromkarte im const. current Modus (Setpoint=\SI{5}{\nano A}, P-Gain=\SI{1000}{}, I-Gain=\SI{2000}{} und Tip voltage=\SI{200}{\nano V})} \label{fig:gr10nm200nVc}
\end{figure}
\begin{figure}[h]
        \centering
        \includegraphics[width=.5\textwidth]{../data/Graphit4_z.png}
        \caption{HOPG: Höhenkarte im const. current Modus (Setpoint=\SI{5}{\nano A}, P-Gain=\SI{1000}{}, I-Gain=\SI{2000}{} und Tip voltage=\SI{200}{\nano V})} \label{fig:gr10nm200nVz}
\end{figure}
\begin{figure}[h]
        \centering
        \includegraphics[width=.5\textwidth]{../data/Graphit5_current.png}
        \caption{HOPG: Stromkarte im const. current Modus (Setpoint=\SI{1}{\nano A}, P-Gain=\SI{1000}{}, I-Gain=\SI{2000}{} und Tip voltage=\SI{50}{\nano V})} \label{fig:gr2nm50nVc}
\end{figure}
\begin{figure}[h]
        \centering
        \includegraphics[width=.5\textwidth]{../data/Graphit5_z.png}
        \caption{HOPG: Höhenkarte im const. current Modus (Setpoint=\SI{1}{\nano A}, P-Gain=\SI{1000}{}, I-Gain=\SI{2000}{} und Tip voltage=\SI{50}{\nano V})} \label{fig:gr2nm50nVz}
\end{figure}
\begin{figure}[h]
        \centering
        \includegraphics[width=.5\textwidth]{../data/Graphit6_current.png}
        \caption{HOPG: Stromkarte im const. current Modus (Setpoint=\SI{1}{\nano A}, P-Gain=\SI{1000}{}, I-Gain=\SI{2000}{} und Tip voltage=\SI{50}{\nano V})} \label{fig:gr2nm50nVc2}
\end{figure}
\begin{figure}[h]
        \centering
        \includegraphics[width=.5\textwidth]{../data/Graphit6_z.png}
        \caption{HOPG: Höhenkarte im const. current Modus (Setpoint=\SI{1}{\nano A}, P-Gain=\SI{1000}{}, I-Gain=\SI{2000}{} und Tip voltage=\SI{50}{\nano V})} \label{fig:gr2nm50nVz2}
\end{figure}
\begin{figure}[h]
        \centering
        \includegraphics[width=.5\textwidth]{../data/Graphit7_current.png}
        \caption{HOPG: Stromkarte im const. current Modus (Setpoint=\SI{1}{\nano A}, P-Gain=\SI{1000}{}, I-Gain=\SI{2000}{} und Tip voltage=\SI{50}{\nano V})} \label{fig:gr2nm50nVc3}
\end{figure}
\begin{figure}[h]
        \centering
        \includegraphics[width=.5\textwidth]{../data/Graphit7_z.png}
        \caption{HOPG: Höhenkarte im const. current Modus (Setpoint=\SI{1}{\nano A}, P-Gain=\SI{0}{}, I-Gain=\SI{4}{} und Tip voltage=\SI{50}{\nano V})} \label{fig:gr2nm50nVz3}
\end{figure}

\end{document}
