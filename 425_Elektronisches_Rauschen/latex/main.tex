\RequirePackage{amsthm} %https://tex.stackexchange.com/questions/687324/unknown-theoremstyle-warning-with-springer-nature-template
\documentclass[sn-mathphys-num,iicol]{sn-jnl}

%\usepackage{sn-jnl.sty}
\usepackage{graphicx}%
\usepackage{multirow}%
\usepackage{amsmath,amssymb,amsfonts}%
\usepackage{amsthm}%
\usepackage{physics}
\usepackage{siunitx}
\usepackage{mathrsfs}%
\usepackage[title]{appendix}%
\usepackage{xcolor}%
\usepackage{textcomp}%
\usepackage{manyfoot}%
\usepackage{booktabs}%
\usepackage{algorithm}%
\usepackage{algorithmicx}%
\usepackage{algpseudocode}%
\usepackage{listings}%
\usepackage{newtxmath}%
\usepackage[tiny]{titlesec}%
\usepackage[ngerman]{babel}
\usepackage{booktabs}

\theoremstyle{thmstyleone}
\newtheorem{theorem}{Theorem}
\newtheorem{proposition}[theorem]{Proposition}

\theoremstyle{thmstyletwo}
\newtheorem{remark}{Remark}

\theoremstyle{thmstylethree}
\newtheorem{definition}{Definition}

\raggedbottom

\newcommand{\td}{\text{d}}

\titleformat{\subsection}{}{\thesubsection}{1em}{\itshape}
\titleformat{\subsubsection}{}{\thesubsubsection}{1em}{\itshape}

\begin{document}
        
\title[]{Praktikum 4 -- Versuch 425: Elektronisches Rauschen}
\author*[1]{\fnm{Jonas} \sur{Wortmann}}\email{s02jwort@uni-bonn.de}
\author*[1]{\fnm{Angelo} \sur{Brade}}\email{s72abrad@uni-bonn.de}
\affil*[1]{Rheinische Friedrich--Wilhelms--Universität, Bonn}

\maketitle

\section{Einleitung}
In jedem elektrischen Schaltkreis ist ein elektronisches Rauschen vorhanden.
Dieses Rauschen setzt sich aus dem \textsc{Johnson}--Rauschen und dem Schrotrauschen zusammen.
Das \textsc{Johnson}--Rauschen ist temperaturabhängig, daraus lässt sich die \textsc{Boltzmann}--Konstante bestimmen.
Das Schrotrauschen wird durch die Quantelung der Elementarladung hervorgerufen. 
Die Größe der Elementarladung lässt sich damit bestimmen.

\section{Bandbreite}
\subsection{Theoretischer Hintergrund}
Die Bandbreite gibt die Breite eines Frequenzsbands an.
Da das elektronische Rauschen von der Bandbreite abhängig ist, wird diese mit einem Bandpass vorgegeben.
Ein Frequenzband einer bestimmten Breite kann mit einem Hoch-- und Tiefpass in Serie (auch Bandpass) erzeugt werden.
Die im Versuch verwendete Anordnung ist in Abb.\ (\ref{fig:schaltplan_bandpass}) gezeigt.

\begin{figure}[t]
        \centering
        \includegraphics[width=.5\textwidth]{425_schaltplan_bandpass.png}
        \caption{Schaltplan des Bandpass.\cite{anleitung425}} \label{fig:schaltplan_bandpass}
\end{figure}

Die effektive Bandbreite wird mit der Verstärkung 
\begin{align} 
        G(f)=\dfrac{V^\textsc{rms}_\text{output}}{V^\textsc{rms}_\text{input}}
\end{align}
bestimmt
\begin{align} 
        \Delta f_\text{eff}=\int_{}^{}\td fG^2(f)=\int_{}^{}\td fG_\text{LP}^2(f)G_\text{HP}^2(f)
,\end{align} 
mit der Tiefpassverstärkung $G_\text{LP}(f)=\left(1+(f/f_\text{l})^4\right)^{-1/2}$ und der Hochpassverstärkung $g_\text{HP}\left(f\right)=\left(f/f_\text{h}\right)^2\left(1+(f/f_\text{h})^4\right)^{-1/2}$.
$f_\text{h}$ und $f_\text{l}$ sind die eingestellten Grenzfrequenzen von Hoch-- und Tiefpass.

\subsection{Durchführung \& Auswertung}
Die Bandbreite wird für $f_\text{h}=\SI{1}{kHz}$ und $f_\text{l}=\SI{10}{kHz}$ bestimmt.
Mit dem Frequenzgenerator werden, für eine konstante Eingangsspannung, verschiedene Frequenzen von $\SI{2}{Hz}$ bis $\SI{8}{MHz}$ eingestellt und die Ausgangsspannung gemessen.
Daraus bestimmt sich jeweils die Verstärkung und damit die Bandbreite.

\iffalse\begin{figure}[t]
        \centering
        \resizebox{.5\textwidth}{!}{\input{eff_band.tex}}
        \caption{Effektive Bandbreite.}
\end{figure}\fi 
%TODO file not found error wtf

\section{\textsc{Johnson}--Rauschen}
\subsection{Theoretischer Hintergrund}
Das \textsc{Johnson}--Rauschen\footnote{auch thermisches Rauschen} entsteht durch thermodynamische Fluktuationen der Elektronen im Leitungsband. 
Dies geschieht, im Vergleich zum Schortrauschen, ohne, dass eine Spannung angelegt ist
Elektronen bewegen sich im thermischen Gleichgewicht ungeordnet aufgrund ihrer thermischen Energie wodurch sie kurze Spannungs-- bzw.\ Strompulse erzeugen.
Mit Hilfe von sensitiven Messgeräten kann dieses Rauschen untersucht werden.

Der formale Zusammenhang der mittleren quadratischen Rauschspannung in einem Widerstand $R$ ist
\begin{align} 
        \overline{V^2}=4k_\text{B}TR\Delta f
,\end{align} 
mit der Temperatur $T$ und der Bandbreite $\Delta f$.  

\subsection{Durchführung \& Auswertung: Beobachtung des \textsc{Johnson}--Rauschens}
Das \textsc{Johnson}--Rauschen wird im Vorverstärkerschaltkreis aus Abb.\ (\ref{fig:vorverstärker}) beobachtet.
Diese Schaltung befindet sich in der LLE--Box Abb.\ (\ref{fig:johnson_lle}), welche entsprechend verkabelt werden muss.
Zur Beobachtung wurden folgende Werte eingestellt
\begin{align} 
        R_\text{in}=\SI{100}{k\ohm},R_\text{f}=\SI{1}{k\ohm}
.\end{align} 
Die LLE--Box wurde dann mit der HLE--Box Abb.\ (\ref{fig:johnson_hle}) verbunden.
Die Einstellung der HLE--Box war
\begin{align} 
        f_\text{h}=\SI{.1}{kHz},f_\text{l}=\SI{100}{kHz},\text{Gain}=300,\text{AC}
.\end{align} 

\begin{figure}[t]
        \centering
        \includegraphics[width=.5\textwidth]{425_schaltplan_vorverstärker_LLE.png}
        \caption{Vorverstärkerschaltkreis zur Beobachtung des \textsc{Johnson}--Rauschen.\cite{anleitung425}} \label{fig:vorverstärker}
\end{figure}

\begin{figure}[t]
        \centering
        \includegraphics[width=.5\textwidth]{425_schaltplan_johnson_LLE.png}
        \caption{Die LLE--Box zur Messung und Beobachtung des \textsc{Johnson}--Rauschen.\cite{anleitung425}} \label{fig:johnson_lle}
\end{figure}

\begin{figure}[t]
        \centering
        \includegraphics[width=.5\textwidth]{425_schaltplan_visualisierung_johnson_HLE.png}
        \caption{HLE--Box zur Messung und Beobachtung des \textsc{Johnson}--Rauschen.\cite{anleitung425}} \label{fig:johnson_hle}
\end{figure}


\begin{figure}[h]
        \centering
        \resizebox{.5\textwidth}{!}{\input{Widerstandsabhängigkeit.tex}}
        \caption{Widerstandsabhängigkeit mit $\overline{V_J^2(t)+V_N^2(t)}=V_{\text{meter}}\frac{\SI{10}{V}}{(600\cdot G_2)^2}$}
\end{figure}
\begin{table}[h!]
    \centering
    \begin{tabular}{cc}
        \textbf{Parameter} & {\textbf{Wert(Fehler)}} \\
        \hline
        m & \SI{1.51 \pm 0.11e-15}{} \\
        b & \SI{7.04 \pm 0.29e-12}{} \\
    \end{tabular}
    \label{tab:parameter}
    \caption{Widerstandsabhängigkeit modelliert mit $\overline{V_J^2(t)+V_N^2(t)}(R)=m\cdot R+b$}
\end{table}

\begin{figure}[h]
        \centering
        \resizebox{.5\textwidth}{!}{\input{Widerstandsresiduen.tex}}
        \caption{$\text{res}=\overline{u^2}-\hat{\overline{u^2}}$}
\end{figure}


\begin{figure}[h]
        \centering
        \resizebox{.5\textwidth}{!}{\input{eff_band.tex}}
        \caption{$G=\frac{U_{\text{RMS}}}{U_{0\text{, RMS}}}$}
\end{figure}
\begin{table}[h!]
    \centering
    \begin{tabular}{cc}
        \textbf{Parameter} & {\textbf{Wert(Fehler)}} \\
        \hline
        $f_l$ & \SI{10104 \pm 56}{} \\
        $f_h$ & \SI{100.50 \pm 0.17}{} \\
    \end{tabular}
    \label{tab:parameter}
    \caption{Frequenzabhängigkeit modelliert mit $G_\text{LP}(f)=\left[1+(f/f_l)^4\right]^{-1/2}$ und $G_\text{HP}(f)=(f/f_h)^2\left[1+(f/f_h)^4\right]^{-1/2}$}
\end{table}

\begin{figure}[h]
        \centering
        \resizebox{.5\textwidth}{!}{\input{Bandbreitenabhängigkeit.tex}}
        \caption{Bandbreitenabhängkeit mit $\overline{V_J^2(t)+V_N^2(t)}=V_{\text{meter}}\frac{\SI{10}{V}}{(600\cdot G_2)^2}$ und $\Delta f_{\text{eff}}=\frac{f_l^4\pi (f_l-f_h)}{2^{3/2}(f_l^4-f_h^4)}$}
\end{figure}
\begin{table}[h!]
    \centering
    \begin{tabular}{cc}
        \textbf{Parameter} & {\textbf{Wert(Fehler)}} \\
        \hline
        m & \SI{7.943 \pm 0.069e-17}{} \\
        b & \SI{7.5 \pm 4.5e-15}{} \\
    \end{tabular}
    \label{tab:parameter}
    \caption{Bandbreitenabhängigkeit modelliert mit $\overline{V_J^2(t)+V_N^2(t)}(\Delta f_{\text{eff}})=m\cdot \Delta f_{\text{eff}}+b$}
\end{table}

\begin{figure}[h]
        \centering
        \resizebox{.5\textwidth}{!}{\input{idc.tex}}
        \caption{Stromabhängigkeit mit $\overline{\delta i^2}=\frac{V_{\text{meter}}\SI{10}{V}}{(100G_2R_f)^2}$ und $i_{\text{dc}}=-\frac{V_{\text{monitor}}}{R_f}$}
\end{figure}
\begin{table}[h!]
    \centering
    \begin{tabular}{cc}
        \textbf{Parameter} & {\textbf{Wert(Fehler)}} \\
        \hline
        m & \SI{4.049 \pm 0.038e-14}{} \\
        b & \SI{2.690 \pm 0.087e-19}{} \\
    \end{tabular}
    \label{tab:parameter}
    \caption{Stromabhängigkeit modelliert mit $\overline{\delta i^2}(i_{\text{dc}})=m\cdot i_{\text{dc}}+b$}
\end{table}

\begin{figure}[h]
        \centering
        \resizebox{.5\textwidth}{!}{\input{shot_freq.tex}}
        \caption{Frequenzabhängigkeit mit $\overline{\delta i^2}=\frac{V_{\text{meter}}\SI{10}{V}}{(100G_2R_f)^2}$}
\end{figure}
\begin{table}[h!]
    \centering
    \begin{tabular}{cc}
        \textbf{Parameter} & {\textbf{Wert(Fehler)}} \\
        \hline
        m & \SI{2.784 \pm 0.056e-23}{} \\
        b & \SI{-2.2 \pm 3.0e-21}{} \\
    \end{tabular}
    \label{tab:parameter}
    \caption{Frequenzabhängigkeit modelliert mit $\overline{\delta i^2}(f)=m\cdot f+b$}
\end{table}
\section{Schortrauschen}
\subsection{Theoretischer Hintergrund}


\bibliography{refs}

\end{document}
